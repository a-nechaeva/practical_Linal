\documentclass[a5paper, 10pt]{article}

% Текст
\usepackage[utf8]{inputenc} % UTF-8 кодировка
\usepackage[russian]{babel} % Русский язык
\usepackage{indentfirst} % красная строка в первом параграфе в главе
% Отображение страниц
\usepackage{geometry} % размеры листа и отступов
\usepackage{listings}
\usepackage{color}

\geometry{
	left=12mm,
	top=25mm,
	right=15mm,
	bottom=17mm,
	marginparsep=0mm,
	marginparwidth=0mm,
	headheight=10mm,
	headsep=7mm,
	nofoot}
\usepackage{afterpage,fancyhdr} % настройка колонтитулов
\pagestyle{fancy}
\fancypagestyle{style}{ % создание нового стиля style
	\fancyhf{} % очистка колонтитулов
	\fancyhead[LO, RE]{Лабораторная работа № 3 } % название документа наверху
	\fancyhead[RO, LE]{ Матрицы в 3D-графике} % название section наверху
	\fancyfoot[RO, LE]{\thepage} % номер страницы справа внизу на нечетных и слева внизу на четных
	\renewcommand{\headrulewidth}{0.25pt} % толщина линии сверху
	\renewcommand{\footrulewidth}{0pt} % толцина линии снизу
}
\fancypagestyle{plain}{ % создание нового стиля plain -- полностью пустого
	\fancyhf{}
	\renewcommand{\headrulewidth}{0pt}
}
\fancypagestyle{title}{ % создание нового стиля title -- для титульной страницы
	\fancyhf{}
	\fancyhead[C]{{\footnotesize
			Министерство образования и науки Российской Федерации\\
			Федеральное государственное автономное образовательное учреждение высшего образования
	}}
	\fancyfoot[C]{{\large 
			Санкт-Петербург, 2023-2024
	}}
	\renewcommand{\headrulewidth}{0pt}
}

% Математика
\usepackage{amsmath, amsfonts, amssymb, amsthm} % Набор пакетов для математических текстов
%\usepackage{dmvnbase} % мехматовский пакет latex-сокращений
\usepackage{cancel} % зачеркивание для сокращений
% Рисунки и фигуры
\usepackage[pdftex]{graphicx} % вставка рисунков
\usepackage{wrapfig, subcaption} % вставка фигур, обтекая текст
\usepackage{caption} % для настройки подписей
\captionsetup{figurewithin=none,labelsep=period, font={small,it}} % настройка подписей к рисункам
% Рисование
\usepackage{tikz} % рисование
\usepackage{circuitikz}
\usepackage{pgfplots} % графики
% Таблицы
\usepackage{multirow} % объединение строк
\usepackage{multicol} % объединение столбцов
% Остальное
\usepackage[unicode, pdftex]{hyperref} % гиперссылки
\usepackage{enumitem} % нормальное оформление списков
\setlist{itemsep=0.15cm,topsep=0.15cm,parsep=1pt} % настройки списков
% Теоремы, леммы, определения...
\theoremstyle{definition}
\newtheorem{Def}{Определение}
\newtheorem*{Axiom}{Аксиома}
\theoremstyle{plain}
\newtheorem{Th}{Теорема}
\newtheorem{Lem}{Лемма}
\newtheorem{Cor}{Следствие}
\newtheorem{Ex}{Пример}
\theoremstyle{remark}
\newtheorem*{Note}{Замечание}
\newtheorem*{Solution}{Решение}
\newtheorem*{Proof}{Доказательство}
% Свои команды
\newcommand{\comb}[1]{\left[\hspace{-4pt}\begin{array}{l}#1\end{array}\right.\hspace{-5pt} } % совокупность уравнений
% Титульный лист
\usepackage{csvsimple-l3}
\newcommand*{\titlePage}{
	\thispagestyle{title}
	\begingroup
	\begin{center}
		%		{\footnotesize
			%			Министерство образования и науки Российской Федерации\\
			%			Федеральное государственное автономное образовательное учреждение высшего образования
			%		}
		%		
		\vspace*{6ex}
		
		{\small
			САНКТ-ПЕТЕРБУРГСКИЙ НАЦИОНАЛЬНЫЙ ИССЛЕДОВАТЕЛЬСКИЙ УНИВЕРСИТЕТ ИТМО	
		}
		
		\vspace*{2ex}
		
		{\normalsize
			Факультет систем управления и робототехники
		}
		
		\vspace*{15ex}
		
		{\Large \bfseries 
			Лабораторная работа № 3
		}
\vspace*{2ex}
	{\Large \bfseries 
			
"Матрицы в 3D-графике "
		}
\vspace*{2ex}
		
		{\normalsize
			по дисциплине Практическая линейная алгебра
		}

	\end{center}
	\vspace*{20ex}
	\begin{flushright}
		{\large 
			\underline{Выполнила}: студентка гр. \textbf{R3238}\\
			\begin{flushright}
				\textbf{Нечаева А. А.}\\
			\end{flushright}
		}
		
		\vspace*{5ex}
		
		{\large 
			\underline{Преподаватель}: \textit{Перегудин Алексей Алексеевич}
		}
	\end{flushright}	
	\newpage
	\setcounter{page}{1}
	\endgroup}

\begin{document}
	\titlePage
	\pagestyle{style}
\newpage




\section{Задание. Создайте кубик.}

\subsection{Как работает код?}
В первой части кода (рисунок 1) задаются координаты вершин куба: каждый столбец -- вершина и сверху вниз в нем заданы координаты $x$, $y$, $z$ в пространстве и $w$ (последняя отвечает за перспективу).
\begin{figure}[h!]
\center{\includegraphics[width=0.5\linewidth]{pic/code1.png}}
\caption{Исходный код кубика, часть 1.}
\end{figure}

Вторая часть (рисунок 2) отвечает за задание плоскостей граней куба, в каждой строчке записаны 4 вершины куба, по которым строится грань.
\begin{figure}[h!]
\center{\includegraphics[width=0.5\linewidth]{pic/code2.png}}
\caption{Исходный код кубика, часть 2.}
\end{figure}

Функция $DrawShape$ отвечает за отрисовку кубика, сначала строятся точки вершин по 3 координатам и с учетом перспективы, затем изображаются грани.
\begin{figure}[h!]
\center{\includegraphics[width=1\linewidth]{pic/code3.png}}
\caption{Исходный код кубика, часть 3.}
\end{figure}

\newpage
\subsection{Почему используется четырехкомпонентный вектор, а не трех?}
Четвертый компонент в векторе позволяет реализовывать перспективную проекцию, а не только отображать ортогональную проекцию. Кроме того, с помощью матрицы $4 \times 4$ реализуются такие преобразования как сдвиги, повороты и т.д. в трехмерном пространстве.

\subsection{Как задать другие фигуры?}






\newpage
\section{Задание. Изменить масштаб кубика.}
Для изменения масштаба кубика использовалась матрица вида:
\begin{equation}
\begin{bmatrix}
a_1 & 0 & 0 & 0\\
0 & a_2 & 0 & 0 \\
0 & 0 & a_3 & 0\\
0 & 0 & 0 & 1
\end{bmatrix}
\end{equation}

Где, $a_1$, $a_2$, $a_3$ отвечают за изменение масштаба по $x$, $y$ и $z$ соответственно.
\begin{figure}[h!]
\center{\includegraphics[width=1\linewidth]{pic/task_2_1.png}}
\caption{Оригинальный масштаб,при $a_i = 1$ .}
\end{figure}

\begin{figure}[h!]
\center{\includegraphics[width=0.75\linewidth]{pic/task_2_2.png}}
\caption{Результат при $a_i = 2$ .}
\end{figure}

\newpage
\begin{figure}[h!]
\center{\includegraphics[width=0.75\linewidth]{pic/task_2_3.png}}
\caption{Результат при $a_i = 0.5$ .}
\end{figure}

\newpage
\begin{figure}[h!]
\center{\includegraphics[width=0.75\linewidth]{pic/task_2_4.png}}
\caption{Результат при $a_i = 5$ .}
\end{figure}

\begin{center}
\begin{lstlisting}
sizeMatrix = [
    5, 0, 0, 0;
    0, 5, 0, 0;
    0, 0, 5, 0;
    0, 0, 0, 1
    ];

newVertices = sizeMatrix * verticesCube;

DrawShape (newVertices, facesCube, 'y')
\end{lstlisting}
\textit{Листинг 1. Часть кода, отвечающая за масшабирование кубика.}
\end{center}



\newpage
\section{Задание. Переместить кубик.}




\newpage
\section{Задание. Выполнить вращение кубика.}



\newpage
\section{Задание. Выполнить вращение кубика около одной вершины.}



\newpage
\section{Задание. Реализация камеры.}


\newpage
\section{Задание. Реализация перспективы.}



\newpage
\section{Задание. * Почти Minecraft.}



Для визуализации был написан код на языке \textit{Python} с использованием библиотек \textit{Matplotlib} и \textit{Numpy}. \\
Код расположен на \href{https://github.com/a-nechaeva/practical_Linal/tree/main/lab2/py_code}{\textbf{GitHub}}.
\\
\textit{Отражение (симметрию) плоскости относительно прямой $y=ax$, в нашем случае после подстановки $a=2$, получаем $y=2x$. Задача -- найти матрицу вида}:


\end{document}













