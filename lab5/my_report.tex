\documentclass[a5paper, 10pt]{article}

% Текст
\usepackage[utf8]{inputenc} % UTF-8 кодировка
\usepackage[russian]{babel} % Русский язык
\usepackage{indentfirst} % красная строка в первом параграфе в главе
% Отображение страниц
\usepackage{geometry} % размеры листа и отступов
\usepackage{listings}
\usepackage{color}

\geometry{
	left=12mm,
	top=25mm,
	right=15mm,
	bottom=17mm,
	marginparsep=0mm,
	marginparwidth=0mm,
	headheight=10mm,
	headsep=7mm,
	nofoot}
\usepackage{afterpage,fancyhdr} % настройка колонтитулов
\pagestyle{fancy}
\fancypagestyle{style}{ % создание нового стиля style
	\fancyhf{} % очистка колонтитулов
	\fancyhead[LO, RE]{Лабораторная работа № 4 } % название документа наверху
	\fancyhead[RO, LE]{Динамические системы} % название section наверху
	\fancyfoot[RO, LE]{\thepage} % номер страницы справа внизу на нечетных и слева внизу на четных
	\renewcommand{\headrulewidth}{0.25pt} % толщина линии сверху
	\renewcommand{\footrulewidth}{0pt} % толцина линии снизу
}
\fancypagestyle{plain}{ % создание нового стиля plain -- полностью пустого
	\fancyhf{}
	\renewcommand{\headrulewidth}{0pt}
}
\fancypagestyle{title}{ % создание нового стиля title -- для титульной страницы
	\fancyhf{}
	\fancyhead[C]{{\footnotesize
			Министерство образования и науки Российской Федерации\\
			Федеральное государственное автономное образовательное учреждение высшего образования
	}}
	\fancyfoot[C]{{\large 
			Санкт-Петербург, 2023-2024
	}}
	\renewcommand{\headrulewidth}{0pt}
}

% Математика
\usepackage{amsmath, amsfonts, amssymb, amsthm} % Набор пакетов для математических текстов
%\usepackage{dmvnbase} % мехматовский пакет latex-сокращений
\usepackage{cancel} % зачеркивание для сокращений
% Рисунки и фигуры
\usepackage[pdftex]{graphicx} % вставка рисунков
\usepackage{wrapfig, subcaption} % вставка фигур, обтекая текст
\usepackage{caption} % для настройки подписей
\captionsetup{figurewithin=none,labelsep=period, font={small,it}} % настройка подписей к рисункам
% Рисование
\usepackage{tikz} % рисование
\usepackage{circuitikz}
\usepackage{pgfplots} % графики
% Таблицы
\usepackage{multirow} % объединение строк
\usepackage{multicol} % объединение столбцов
% Остальное
\usepackage[unicode, pdftex]{hyperref} % гиперссылки
\usepackage{enumitem} % нормальное оформление списков
\setlist{itemsep=0.15cm,topsep=0.15cm,parsep=1pt} % настройки списков
% Теоремы, леммы, определения...
\theoremstyle{definition}
\newtheorem{Def}{Определение}
\newtheorem*{Axiom}{Аксиома}
\theoremstyle{plain}
\newtheorem{Th}{Теорема}
\newtheorem{Lem}{Лемма}
\newtheorem{Cor}{Следствие}
\newtheorem{Ex}{Пример}
\theoremstyle{remark}
\newtheorem*{Note}{Замечание}
\newtheorem*{Solution}{Решение}
\newtheorem*{Proof}{Доказательство}
% Свои команды
\newcommand{\comb}[1]{\left[\hspace{-4pt}\begin{array}{l}#1\end{array}\right.\hspace{-5pt} } % совокупность уравнений
% Титульный лист
\usepackage{csvsimple-l3}
\newcommand*{\titlePage}{
	\thispagestyle{title}
	\begingroup
	\begin{center}
		%		{\footnotesize
			%			Министерство образования и науки Российской Федерации\\
			%			Федеральное государственное автономное образовательное учреждение высшего образования
			%		}
		%		
		\vspace*{6ex}
		
		{\small
			САНКТ-ПЕТЕРБУРГСКИЙ НАЦИОНАЛЬНЫЙ ИССЛЕДОВАТЕЛЬСКИЙ УНИВЕРСИТЕТ ИТМО	
		}
		
		\vspace*{2ex}
		
		{\normalsize
			Факультет систем управления и робототехники
		}
		
		\vspace*{15ex}
		
		{\Large \bfseries 
			Лабораторная работа № 5
		}
\vspace*{2ex}
	{\Large \bfseries 
			
"Спектральная теория графов"
		}
\vspace*{2ex}
		
		{\normalsize
			по дисциплине Практическая линейная алгебра
		}

	\end{center}
	\vspace*{20ex}
	\begin{flushright}
		{\large 
			\underline{Выполнила}: студентка гр. \textbf{R3238}\\
			\begin{flushright}
				\textbf{Нечаева А. А.}\\
			\end{flushright}
		}
		
		\vspace*{5ex}
		
		{\large 
			\underline{Преподаватель}: \textit{Перегудин Алексей Алексеевич}
		}
	\end{flushright}	
	\newpage
	\setcounter{page}{1}
	\endgroup}

\begin{document}
	\titlePage
	\pagestyle{style}
\newpage
\section{Кластеризация социальной сети}
Для начала построим модель небольшой социальной сети, где каждый пользователь обозначен одной из 18 вершин графа, а ребра показывают дружбу между людьми.

\begin{figure}[h!]
\center{\includegraphics[width=1\linewidth]{pic/gr.png}}
\caption{Модель социальной небольшой сети}
\end{figure}
Соотвествующая графу на рисунке 1 матрица Лапласа:
\begin{figure}[h!]
\center{\includegraphics[width=1\linewidth]{pic/lap_1.png}}
\caption{Матрица Лапласа}
\end{figure}
Все собственные числа и соответствующие им собственные векторы приведены в разделе \textit{Приложение} в конце документа.\\

Ниже приведены все собственные векторы и соотвествующие им собственные числа

\newpage
Для визуализации был написан код на языке \textit{Python}. \\
Код расположен на \href{https://github.com/a-nechaeva/practical_Linal/tree/main/lab4}{\textbf{GitHub}}.
\section{Приложение}
Все собственные числа соответствующие им собственные векторы для матрицы Лапласа из задания 1.
\begin{equation*}
\left(\begin{matrix}
1 \\
1 \\
1 \\
1 \\
1 \\
1 \\
1 \\
1 \\
1 \\
1 \\
1 \\
1 \\
1 \\
1 \\
1 \\
1 \\
1 \\
1
\end{matrix}\right)
 , \, \lambda_ 1 = 0\,\,\,\,\,\,\,\,\,\,
\left(\begin{matrix}
0 \\
0 \\
0 \\
0 \\
0 \\
0 \\
-1 \\
1 \\
0 \\
0 \\
0 \\
0 \\
0 \\
0 \\
0 \\
0 \\
0 \\
0
\end{matrix}\right)
, \, \lambda_ 2 = 3\,\,\,\,\,\,\,\,\,\,
\left(\begin{matrix}
0 \\
0 \\
0 \\
0 \\
0 \\
0 \\
0 \\
0 \\
0 \\
0 \\
0 \\
0 \\
0 \\
0 \\
-1 \\
0 \\
1 \\
0
\end{matrix}\right)
, \, \lambda_ 3 = 4\,\,\,\,\,\,\,\,\,\,
\left(\begin{matrix}
0 \\
0 \\
0 \\
0 \\
0 \\
0 \\
0 \\
0 \\
0 \\
0 \\
0 \\
0 \\
0 \\
0 \\
-1 \\
0 \\
1 \\
0
\end{matrix}\right)
, \, \lambda_ 4 = 4\,\,\,\,\,\,\,\,\,\,
\end{equation*}

\begin{equation*}
\left(\begin{matrix}
\left(-0,541\right) \\
\left(-0,609\right) \\
\left(-0,470\right) \\
\left(-0,329\right) \\
\left(-0,497\right) \\
\left(-0,493\right) \\
\left(-0,933\right) \\
\left(-0,933\right) \\
\left(-0,850\right) \\
\left(0,375\right) \\
\left(0,173\right) \\
\left(0,276\right) \\
\left(0,355\right) \\
\left(0,564\right) \\
1 \\
\left(0,911\right) \\
1 \\
1
\end{matrix}\right)
,  \lambda_ 5 = 0.089\,\,\,\,\,\,\,\,\,\,
\left(\begin{matrix}
\left(-0,569\right) \\
\left(-0,184\right) \\
\left(-0,736\right) \\
\left(-0,803\right) \\
\left(-0,614\right) \\
\left(-0,594\right) \\
\left(1,570\right) \\
\left(1,570\right) \\
\left(1,096\right) \\
\left(-0,885\right) \\
\left(-1,026\right) \\
\left(-1,160\right) \\
\left(-0,944\right) \\
\left(-0,419\right) \\
1 \\
\left(0,698\right) \\
1 \\
1
\end{matrix}\right)
,  \lambda_ 6 = 0.302\,\,\,\,\,\,\,\,\,\,
\left(\begin{matrix}
\left(1,766\right) \\
\left(1,141\right) \\
\left(1,718\right) \\
\left(0,976\right) \\
\left(1,569\right) \\
\left(1,470\right) \\
\left(-1,573\right) \\
\left(-1,573\right) \\
\left(-0,799\right) \\
\left(-1,742\right) \\
\left(-1,333\right) \\
\left(-2,090\right) \\
\left(-1,819\right) \\
\left(-1,217\right) \\
1 \\
\left(0,508\right) \\
1 \\
1
\end{matrix}\right)
,  \lambda_ 7 = 0.492\,\,\,\,\,\,\,\,\,\,
\end{equation*}

\begin{equation*}
\left(\begin{matrix}
\left(-0,483\right) \\
\left(-0,304\right) \\
\left(-0,109\right) \\
\left(0,480\right) \\
\left(-0,117\right) \\
\left(0,055\right) \\
\left(0,102\right) \\
\left(0,102\right) \\
\left(-0,092\right) \\
\left(-3,902\right) \\
\left(1,177\right) \\
\left(6,448\right) \\
\left(-0,559\right) \\
\left(-4,895\right) \\
1 \\
\left(-0,904\right) \\
1 \\
1
\end{matrix}\right)
,  \lambda_ 8 = 1.904\,\,\,\,\,\,\,
\left(\begin{matrix}
\left(-112,670\right) \\
\left(51,599\right) \\
\left(-122,421\right) \\
\left(36,220\right) \\
\left(5,800\right) \\
\left(162,222\right) \\
\left(-18,290\right) \\
\left(-18,290\right) \\
\left(26,025\right) \\
\left(2,964\right) \\
\left(11,522\right) \\
\left(-16,445\right) \\
\left(-4,567\right) \\
\left(-5,244\right) \\
1 \\
\left(-1,423\right) \\
1 \\
1
\end{matrix}\right)
,  \lambda_ 9 = 2.423\,\,\,\,\,\,\,\,
\left(\begin{matrix}
\left(-15,810\right) \\
\left(-19,115\right) \\
\left(13,664\right) \\
\left(19,505\right) \\
\left(0,545\right) \\
\left(3,014\right) \\
\left(7,572\right) \\
\left(7,572\right) \\
\left(-12,795\right) \\
\left(2,437\right) \\
\left(8,335\right) \\
\left(-8,656\right) \\
\left(-2,365\right) \\
\left(-5,214\right) \\
1 \\
\left(-1,690\right) \\
1 \\
1
\end{matrix}\right),  \lambda_ {10} =2.690\,\,\,\,\,\,\,\,\,\,
\end{equation*}

\begin{equation*}
\left(\begin{matrix}
\left(0,661\right) \\
\left(0,984\right) \\
\left(-0,783\right) \\
\left(0,071\right) \\
\left(-0,445\right) \\
\left(-1,660\right) \\
\left(-0,872\right) \\
\left(-0,872\right) \\
\left(2,065\right) \\
\left(3,889\right) \\
\left(2,932\right) \\
\left(-2,241\right) \\
\left(0,133\right) \\
\left(-4,497\right) \\
1 \\
\left(-2,368\right) \\
1 \\
1
\end{matrix}\right)
,  \lambda_ {11} =3.368\,\,\,\,\,\,\,\,
\left(\begin{matrix}
\left(9,817\right) \\
\left(-0,245\right) \\
\left(-8,229\right) \\
\left(-5,286\right) \\
\left(1,399\right) \\
\left(5,733\right) \\
\left(6,255\right) \\
\left(6,255\right) \\
\left(-17,018\right) \\
\left(4,618\right) \\
\left(-0,381\right) \\
\left(-0,251\right) \\
\left(0,812\right) \\
\left(-3,760\right) \\
1 \\
\left(-2,721\right) \\
1 \\
1
\end{matrix}\right)
,  \lambda_ {12} =3.721\,\,\,\,\,\,\,
\left(\begin{matrix}
\left(-3,665\right) \\
\left(-0,297\right) \\
\left(5,560\right) \\
\left(-4,796\right) \\
\left(0,190\right) \\
\left(3,295\right) \\
\left(-0,093\right) \\
\left(-0,093\right) \\
\left(0,324\right) \\
\left(5,299\right) \\
\left(-6,706\right) \\
\left(2,647\right) \\
\left(0,121\right) \\
\left(-1,299\right) \\
1 \\
\left(-3,488\right) \\
1 \\
1
\end{matrix}\right)
,  \lambda_ {13} =4.488\,\,\,\,\,\,\,\,\,\,
\end{equation*}

\begin{equation*}
\left(\begin{matrix}
\left(-0,288\right) \\
\left(0,067\right) \\
\left(0,517\right) \\
\left(-0,629\right) \\
\left(-0,047\right) \\
\left(0,327\right) \\
\left(0,013\right) \\
\left(0,013\right) \\
\left(-0,050\right) \\
\left(-3,826\right) \\
\left(-0,251\right) \\
\left(-2,370\right) \\
\left(7,043\right) \\
\left(0,348\right) \\
1 \\
\left(-3,866\right) \\
1 \\
1
\end{matrix}\right)
,  \lambda_ {14} =4.866\,\,\,\,\,\,\,\,\,\,
\left(\begin{matrix}
\left(1,758\right) \\
\left(-2,275\right) \\
\left(-1,818\right) \\
\left(2,390\right) \\
\left(0,041\right) \\
\left(-0,068\right) \\
\left(-0,287\right) \\
\left(-0,287\right) \\
\left(1,236\right) \\
\left(0,292\right) \\
\left(-1,273\right) \\
\left(0,995\right) \\
\left(-2,016\right) \\
\left(2,616\right) \\
1 \\
\left(-4,305\right) \\
1 \\
1
\end{matrix}\right)
,  \lambda_ {15} = 5.305\,\,\,\,\,\,\,\,\,\,
\left(\begin{matrix}
\left(-6,448\right) \\
\left(10,339\right) \\
\left(-0,417\right) \\
\left(2,122\right) \\
\left(5,312\right) \\
\left(-7,523\right) \\
\left(1,245\right) \\
\left(1,245\right) \\
\left(-5,430\right) \\
\left(-0,064\right) \\
\left(-0,263\right) \\
\left(0,831\right) \\
\left(-2,531\right) \\
\left(2,945\right) \\
1 \\
\left(-4,363\right) \\
1 \\
1
\end{matrix}\right)
,  \lambda_ {16} = 5.363\,\,\,\,\,\,\,\,\,\,
\end{equation*}

\begin{equation*}
\left(\begin{matrix}
\left(-0,141\right) \\
\left(2,993\right) \\
\left(5,426\right) \\
\left(-6,661\right) \\
\left(-8,035\right) \\
\left(4,280\right) \\
\left(0,273\right) \\
\left(0,273\right) \\
\left(-1,295\right) \\
\left(-3,217\right) \\
\left(9,878\right) \\
\left(-0,961\right) \\
\left(-6,291\right) \\
\left(5,209\right) \\
1 \\
\left(-4,734\right) \\
1 \\
1
\end{matrix}\right)
,  \lambda_ {17} = 5.734\,\,\,\,\,\,\,\,\,\,
\left(\begin{matrix}
\left(-13,211\right) \\
\left(-42,779\right) \\
\left(-7,706\right) \\
\left(-55,201\right) \\
\left(93,510\right) \\
\left(1,373\right) \\
\left(-2,829\right) \\
\left(-2,829\right) \\
\left(14,873\right) \\
\left(-10,039\right) \\
\left(37,402\right) \\
\left(-5,598\right) \\
\left(-13,572\right) \\
\left(8,864\right) \\
1 \\
\left(-5,257\right) \\
1 \\
1
\end{matrix}\right)
,  \lambda_ {18} = 6.257\,\,\,\,\,\,\,\,\,\,
\end{equation*}

\end{document}













