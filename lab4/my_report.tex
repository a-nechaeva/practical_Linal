\documentclass[a5paper, 10pt]{article}

% Текст
\usepackage[utf8]{inputenc} % UTF-8 кодировка
\usepackage[russian]{babel} % Русский язык
\usepackage{indentfirst} % красная строка в первом параграфе в главе
% Отображение страниц
\usepackage{geometry} % размеры листа и отступов
\usepackage{listings}
\usepackage{color}

\geometry{
	left=12mm,
	top=25mm,
	right=15mm,
	bottom=17mm,
	marginparsep=0mm,
	marginparwidth=0mm,
	headheight=10mm,
	headsep=7mm,
	nofoot}
\usepackage{afterpage,fancyhdr} % настройка колонтитулов
\pagestyle{fancy}
\fancypagestyle{style}{ % создание нового стиля style
	\fancyhf{} % очистка колонтитулов
	\fancyhead[LO, RE]{Лабораторная работа № 4 } % название документа наверху
	\fancyhead[RO, LE]{Динамические системы} % название section наверху
	\fancyfoot[RO, LE]{\thepage} % номер страницы справа внизу на нечетных и слева внизу на четных
	\renewcommand{\headrulewidth}{0.25pt} % толщина линии сверху
	\renewcommand{\footrulewidth}{0pt} % толцина линии снизу
}
\fancypagestyle{plain}{ % создание нового стиля plain -- полностью пустого
	\fancyhf{}
	\renewcommand{\headrulewidth}{0pt}
}
\fancypagestyle{title}{ % создание нового стиля title -- для титульной страницы
	\fancyhf{}
	\fancyhead[C]{{\footnotesize
			Министерство образования и науки Российской Федерации\\
			Федеральное государственное автономное образовательное учреждение высшего образования
	}}
	\fancyfoot[C]{{\large 
			Санкт-Петербург, 2023-2024
	}}
	\renewcommand{\headrulewidth}{0pt}
}

% Математика
\usepackage{amsmath, amsfonts, amssymb, amsthm} % Набор пакетов для математических текстов
%\usepackage{dmvnbase} % мехматовский пакет latex-сокращений
\usepackage{cancel} % зачеркивание для сокращений
% Рисунки и фигуры
\usepackage[pdftex]{graphicx} % вставка рисунков
\usepackage{wrapfig, subcaption} % вставка фигур, обтекая текст
\usepackage{caption} % для настройки подписей
\captionsetup{figurewithin=none,labelsep=period, font={small,it}} % настройка подписей к рисункам
% Рисование
\usepackage{tikz} % рисование
\usepackage{circuitikz}
\usepackage{pgfplots} % графики
% Таблицы
\usepackage{multirow} % объединение строк
\usepackage{multicol} % объединение столбцов
% Остальное
\usepackage[unicode, pdftex]{hyperref} % гиперссылки
\usepackage{enumitem} % нормальное оформление списков
\setlist{itemsep=0.15cm,topsep=0.15cm,parsep=1pt} % настройки списков
% Теоремы, леммы, определения...
\theoremstyle{definition}
\newtheorem{Def}{Определение}
\newtheorem*{Axiom}{Аксиома}
\theoremstyle{plain}
\newtheorem{Th}{Теорема}
\newtheorem{Lem}{Лемма}
\newtheorem{Cor}{Следствие}
\newtheorem{Ex}{Пример}
\theoremstyle{remark}
\newtheorem*{Note}{Замечание}
\newtheorem*{Solution}{Решение}
\newtheorem*{Proof}{Доказательство}
% Свои команды
\newcommand{\comb}[1]{\left[\hspace{-4pt}\begin{array}{l}#1\end{array}\right.\hspace{-5pt} } % совокупность уравнений
% Титульный лист
\usepackage{csvsimple-l3}
\newcommand*{\titlePage}{
	\thispagestyle{title}
	\begingroup
	\begin{center}
		%		{\footnotesize
			%			Министерство образования и науки Российской Федерации\\
			%			Федеральное государственное автономное образовательное учреждение высшего образования
			%		}
		%		
		\vspace*{6ex}
		
		{\small
			САНКТ-ПЕТЕРБУРГСКИЙ НАЦИОНАЛЬНЫЙ ИССЛЕДОВАТЕЛЬСКИЙ УНИВЕРСИТЕТ ИТМО	
		}
		
		\vspace*{2ex}
		
		{\normalsize
			Факультет систем управления и робототехники
		}
		
		\vspace*{15ex}
		
		{\Large \bfseries 
			Лабораторная работа № 4
		}
\vspace*{2ex}
	{\Large \bfseries 
			
"Динамические системы"
		}
\vspace*{2ex}
		
		{\normalsize
			по дисциплине Практическая линейная алгебра
		}

	\end{center}
	\vspace*{20ex}
	\begin{flushright}
		{\large 
			\underline{Выполнила}: студентка гр. \textbf{R3238}\\
			\begin{flushright}
				\textbf{Нечаева А. А.}\\
			\end{flushright}
		}
		
		\vspace*{5ex}
		
		{\large 
			\underline{Преподаватель}: \textit{Перегудин Алексей Алексеевич}
		}
	\end{flushright}	
	\newpage
	\setcounter{page}{1}
	\endgroup}

\begin{document}
	\titlePage
	\pagestyle{style}
\newpage
\textit{В этой лабораторной мы будем работать с непрерывными $ \left( t \in \mathbb{R} \right) $ и дискретными $ \left( k \in \mathbb{Z} \right)$  линейными динамическими системами второго порядка вида}

\begin{equation}
\begin{cases}
\dot{x}_1(t) = a_1x_1(t)+a_2x_2(t),\\
\dot{x}_2(t) = a_3x_1(t)+a_4x_2(t)
\end{cases}
\end{equation}

\begin{equation}
\begin{cases}
x_1(k+1) = a_1x_1(k)+a_2x_2(k),\\
x_2(k+1) = a_3x_1(k)+a_4x_2(k)
\end{cases}
\end{equation}
в более компактной форме:
\begin{equation}
\dot{x}(t) = Ax(t),
\end{equation}

\begin{equation}
x(k+1) = Ax(k),
\end{equation}
где $x(\cdot) \in  \mathbb{R}^2, \,  \mathbb{R}^{2 \times 2} $.

\newpage
\section{задание}
Зададимся двумя неколлинеарными векторами $v_1, \, v_2 \in \mathbb{R}^2$, не лежащими на координатных осях:
\begin{equation}
v_1 =
\begin{pmatrix}
1\\
4\\
\end{pmatrix}
\, \, \, \, \, \, \, \, \, \,
v_2 =
\begin{pmatrix}
2\\
3\\
\end{pmatrix}
\end{equation}
Придумаем непрерывные динамические системы:

% part 1
\subsection{}
\textit{Система ассимптотически устойчива, при этом если $x(0) = v_1$, то $x(t) \in Span{v_1}$, а если $x(0) = v_2,$ то  $x(t) \in Span{v_2}$ при всех $t \geq 0$.}


% part 2
\subsection{}
\textit{Система неустойчива, при этом у матрицы $A$ не существует двух неколлинеарных собственных векторов.}


% part 3
\subsection{}
\textit{Система неустойчива, при этом если $x(0) = v_1$, то $\lim\limits_{t \to \infty} x(t) = 0$.}


% part 4
\subsection{}
\textit{Система асимптотически устойчива, при этом матрица $A \in \mathbb{R}^2$ имеет комплексные собственные вектора вида $v_1 \pm v_2 i \in \mathbb{C}^2$.}


% part 5
\subsection{}
\textit{Система неустойчива, при этом матрица $A$ имеет такие же слбственные вектора, как в предыдущем пункте.}



% part 6
\subsection{}
\textit{Система не является асимптотически устойчивой, но не является и неустойчивой, при этом матрица $A$ имеет собственные векторы такие же, как в пункте 4.}


\end{document}













