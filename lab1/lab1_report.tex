\documentclass[a5paper, 10pt]{article}

% Текст
\usepackage[utf8]{inputenc} % UTF-8 кодировка
\usepackage[russian]{babel} % Русский язык
\usepackage{indentfirst} % красная строка в первом параграфе в главе
% Отображение страниц
\usepackage{geometry} % размеры листа и отступов
\geometry{
	left=12mm,
	top=25mm,
	right=15mm,
	bottom=17mm,
	marginparsep=0mm,
	marginparwidth=0mm,
	headheight=10mm,
	headsep=7mm,
	nofoot}
\usepackage{afterpage,fancyhdr} % настройка колонтитулов
\pagestyle{fancy}
\fancypagestyle{style}{ % создание нового стиля style
	\fancyhf{} % очистка колонтитулов
	\fancyhead[LO, RE]{Лабораторная работа № 1 } % название документа наверху
	\fancyhead[RO, LE]{ Кодирование и шифрование} % название section наверху
	\fancyfoot[RO, LE]{\thepage} % номер страницы справа внизу на нечетных и слева внизу на четных
	\renewcommand{\headrulewidth}{0.25pt} % толщина линии сверху
	\renewcommand{\footrulewidth}{0pt} % толцина линии снизу
}
\fancypagestyle{plain}{ % создание нового стиля plain -- полностью пустого
	\fancyhf{}
	\renewcommand{\headrulewidth}{0pt}
}
\fancypagestyle{title}{ % создание нового стиля title -- для титульной страницы
	\fancyhf{}
	\fancyhead[C]{{\footnotesize
			Министерство образования и науки Российской Федерации\\
			Федеральное государственное автономное образовательное учреждение высшего образования
	}}
	\fancyfoot[C]{{\large 
			Санкт-Петербург, 2023-2024
	}}
	\renewcommand{\headrulewidth}{0pt}
}

% Математика
\usepackage{amsmath, amsfonts, amssymb, amsthm} % Набор пакетов для математических текстов
%\usepackage{dmvnbase} % мехматовский пакет latex-сокращений
\usepackage{cancel} % зачеркивание для сокращений
% Рисунки и фигуры
\usepackage[pdftex]{graphicx} % вставка рисунков
\usepackage{wrapfig, subcaption} % вставка фигур, обтекая текст
\usepackage{caption} % для настройки подписей
\captionsetup{figurewithin=none,labelsep=period, font={small,it}} % настройка подписей к рисункам
% Рисование
\usepackage{tikz} % рисование
\usepackage{circuitikz}
\usepackage{pgfplots} % графики
% Таблицы
\usepackage{multirow} % объединение строк
\usepackage{multicol} % объединение столбцов
% Остальное
\usepackage[unicode, pdftex]{hyperref} % гиперссылки
\usepackage{enumitem} % нормальное оформление списков
\setlist{itemsep=0.15cm,topsep=0.15cm,parsep=1pt} % настройки списков
% Теоремы, леммы, определения...
\theoremstyle{definition}
\newtheorem{Def}{Определение}
\newtheorem*{Axiom}{Аксиома}
\theoremstyle{plain}
\newtheorem{Th}{Теорема}
\newtheorem{Lem}{Лемма}
\newtheorem{Cor}{Следствие}
\newtheorem{Ex}{Пример}
\theoremstyle{remark}
\newtheorem*{Note}{Замечание}
\newtheorem*{Solution}{Решение}
\newtheorem*{Proof}{Доказательство}
% Свои команды
\newcommand{\comb}[1]{\left[\hspace{-4pt}\begin{array}{l}#1\end{array}\right.\hspace{-5pt} } % совокупность уравнений
% Титульный лист
\usepackage{csvsimple-l3}
\newcommand*{\titlePage}{
	\thispagestyle{title}
	\begingroup
	\begin{center}
		%		{\footnotesize
			%			Министерство образования и науки Российской Федерации\\
			%			Федеральное государственное автономное образовательное учреждение высшего образования
			%		}
		%		
		\vspace*{6ex}
		
		{\small
			САНКТ-ПЕТЕРБУРГСКИЙ НАЦИОНАЛЬНЫЙ ИССЛЕДОВАТЕЛЬСКИЙ УНИВЕРСИТЕТ ИТМО	
		}
		
		\vspace*{2ex}
		
		{\normalsize
			Факультет систем управления и робототехники
		}
		
		\vspace*{15ex}
		
		{\Large \bfseries 
			Лабораторная работа № 1
		}
\vspace*{2ex}
	{\Large \bfseries 
			
"Кодирование и шифрование"
		}
\vspace*{2ex}
		
		{\normalsize
			по дисциплине Практическая линейная алгебра
		}

	\end{center}
	\vspace*{20ex}
	\begin{flushright}
		{\large 
			\underline{Выполнила}: студентка гр. \textbf{R3238}\\
			\begin{flushright}
				\textbf{Нечаева А. А.}\\
			\end{flushright}
		}
		
		\vspace*{5ex}
		
		{\large 
			\underline{Преподаватель}: \textit{Перегудин Алексей Алексеевич}
		}
	\end{flushright}	
	\newpage
	\setcounter{page}{1}
	\endgroup}

\begin{document}
	\titlePage
	\pagestyle{style}
\newpage

\section{Задание 1. Шифр Хилла}
\subsection{Задание алфавита и сообщения}


%\caption{\label{tab:canonsummary}Используемый алфавит .}

\begin{center}
Таблица 1 -- Используемый алфавит\\
\begin{tabular}{ |c|c|c|c|c|c| } 
 \hline
Символ & Код & Символ & Код & Символ & Код\\
\hline
А & 0 & З  & 4 & Ы  & 8 \\
 \hline
В & 1 & Л & 5& Ь  & 9  \\
 \hline
Д & 2 & Н & 6& Я  & 10  \\
 \hline
Ё & 3 & П & 7&   &   \\
 \hline
\end{tabular}
\end{center}

Зашифрованное сообщение: \textbf{\textit{ЗВЁЗДНАЯПЫЛЬ}}

Размер алфавита в нашем случае: $$n = 11 $$

У числа  \textbf{11} нет делителей, кроме единицы и самого числа.

\subsection{Шифрование с помощью матрицы-ключа $2 \times 2$}
Матрица-ключ размера  $2 \times 2$ :
\begin{equation}
A =
\begin{pmatrix}
1 & 2 \\
4 & 9
\end{pmatrix}
\end{equation}
Проверка определителя:
\begin{equation}
\begin{vmatrix}
1 & 2 \\
4 & 9
\end{vmatrix}
= 1 \neq 0
\end{equation}

Запишем фразу, подлежащую шифрования с помощью кодов символов алфавита и разобьем наше сообщение на векторы. \\
Далее представлены фрагменты сообщения и соотвествующие векторы кодов:
\begin{center}
\textbf{\textit{ЗВ}} $\to \begin{pmatrix}
 4\\
1
\end{pmatrix}$ ;
\textbf{\textit{ЁЗ}}  $\to \begin{pmatrix}
 3\\
4
\end{pmatrix}$ ;
\textbf{\textit{ДН}}  $\to \begin{pmatrix}
 2\\
6
\end{pmatrix}$ ;
\textbf{\textit{АЯ}}  $\to \begin{pmatrix}
 0\\
10
\end{pmatrix}$ \\
\textbf{\textit{ПЫ}}  $\to \begin{pmatrix}
 7\\
8
\end{pmatrix}$ ;
\textbf{\textit{ЛЬ}}  $\to \begin{pmatrix}
 5\\
9
\end{pmatrix}$ \\
\end{center}

Теперь зашифруем сообщение: матрично умножим ключ на каждый вектор и найдем остаток от деления на размер алфавита от результата:
\begin{equation}
\begin{pmatrix}
1 & 2 \\
4 & 9
\end{pmatrix}
 \times
\begin{pmatrix}
 4\\
1
\end{pmatrix}
(mod \text{ }11)
= 
\begin{pmatrix}
 6\\
25
\end{pmatrix}
(mod \text{ }11)
= \begin{pmatrix}
 6\\
3
\end{pmatrix}
\end{equation}

\begin{equation}
\begin{pmatrix}
1 & 2 \\
4 & 9
\end{pmatrix}
 \times
\begin{pmatrix}
 3\\
4
\end{pmatrix}
(mod \text{ }11)
= 
\begin{pmatrix}
 11\\
48
\end{pmatrix}
(mod \text{ }11)
= \begin{pmatrix}
 0\\
4
\end{pmatrix}
\end{equation}

\begin{equation}
\begin{pmatrix}
1 & 2 \\
4 & 9
\end{pmatrix}
 \times
\begin{pmatrix}
 2\\
6
\end{pmatrix}
(mod \text{ }11)
= 
\begin{pmatrix}
 14\\
62
\end{pmatrix}
(mod \text{ }11)
= \begin{pmatrix}
 3\\
7
\end{pmatrix}
\end{equation}

\begin{equation}
\begin{pmatrix}
1 & 2 \\
4 & 9
\end{pmatrix}
 \times
\begin{pmatrix}
 0\\
10
\end{pmatrix}
(mod \text{ }11)
= 
\begin{pmatrix}
 20\\
90
\end{pmatrix}
(mod \text{ }11)
= \begin{pmatrix}
 9\\
2
\end{pmatrix}
\end{equation}

\begin{equation}
\begin{pmatrix}
1 & 2 \\
4 & 9
\end{pmatrix}
 \times
\begin{pmatrix}
 7\\
8
\end{pmatrix}
(mod \text{ }11)
= 
\begin{pmatrix}
 23\\
100
\end{pmatrix}
(mod \text{ }11)
= \begin{pmatrix}
 1\\
1
\end{pmatrix}
\end{equation}

\begin{equation}
\begin{pmatrix}
1 & 2 \\
4 & 9
\end{pmatrix}
 \times
\begin{pmatrix}
 5\\
9
\end{pmatrix}
(mod \text{ }11)
= 
\begin{pmatrix}
 23\\
101
\end{pmatrix}
(mod \text{ }11)
= \begin{pmatrix}
 1\\
2
\end{pmatrix}
\end{equation}

Декодируем полученный результат:
\begin{center}
 $ \begin{pmatrix}
 6\\
3
\end{pmatrix} \to$ \textbf{\textit{НЁ}} ;
 $ \begin{pmatrix}
 0\\
4
\end{pmatrix} \to$ \textbf{\textit{АЗ}} ;
 $ \begin{pmatrix}
 3\\
7
\end{pmatrix} \to$ \textbf{\textit{ЁП}} ;
 $ \begin{pmatrix}
9\\
2
\end{pmatrix} \to$ \textbf{\textit{ЬД}} ; \\
 $ \begin{pmatrix}
 1\\
1
\end{pmatrix} \to$ \textbf{\textit{ВВ}} ;
$\begin{pmatrix}
 1\\
2
\end{pmatrix} \to$ \textbf{\textit{ВД}}  \\
\end{center}
Полученное сообщение:  \textbf{\textit{НЁАЗЁПЬДВВВД}}


\subsection{Шифрование с помощью матрицы-ключа $3 \times 3$}
Матрица-ключ размера  $3 \times 3$ :
\begin{equation}
B =
\begin{pmatrix}
1 & 1 & 0 \\
0 & 0 & 1\\
1 & 0 & 1
\end{pmatrix}
\end{equation}
Проверка определителя:
\begin{equation}
\begin{vmatrix}
1 & 1 & 0 \\
0 & 0 & 1\\
1 & 0 & 1
\end{vmatrix}
= 1 \neq 0
\end{equation}
Разобьем сообщение на фрагменты длины 3 и запишем соотвествующие им векторы кодов:
\begin{center}
\textbf{\textit{ЗВЁ}} $\to \begin{pmatrix}
 4\\
1\\
3
\end{pmatrix}$ ;

\textbf{\textit{ЗДН}}  $\to \begin{pmatrix}
4\\
 2\\
6
\end{pmatrix}$ ;
\textbf{\textit{АЯП}}  $\to \begin{pmatrix}
 0\\
10\\
7
\end{pmatrix}$ ;

\textbf{\textit{ЫЛЬ}}  $\to \begin{pmatrix}
8\\
 5\\
9
\end{pmatrix}$ \\
\end{center}
Повторяем действия, описанные в разделе 1.2:
\begin{equation}
\begin{pmatrix}
 1 & 1 & 0 \\
0 & 0 & 1\\
1 & 0 & 1
\end{pmatrix}
 \times
\begin{pmatrix}
 4\\
1\\
3
\end{pmatrix}
(mod \text{ }11)
= 
\begin{pmatrix}
 5\\
3\\
7
\end{pmatrix}
(mod \text{ }11)
= \begin{pmatrix}
5 \\
3\\
7
\end{pmatrix}
\end{equation}

\begin{equation}
\begin{pmatrix}
 1 & 1 & 0 \\
0 & 0 & 1\\
1 & 0 & 1
\end{pmatrix}
 \times
\begin{pmatrix}
 4\\
2\\
6
\end{pmatrix}
(mod \text{ }11)
= 
\begin{pmatrix}
 6\\
6\\
10
\end{pmatrix}
(mod \text{ }11)
= \begin{pmatrix}
6 \\
6\\
10
\end{pmatrix}
\end{equation}

\begin{equation}
\begin{pmatrix}
 1 & 1 & 0 \\
0 & 0 & 1\\
1 & 0 & 1
\end{pmatrix}
 \times
\begin{pmatrix}
 0\\
10\\
7
\end{pmatrix}
(mod \text{ }11)
= 
\begin{pmatrix}
 10\\
7\\
7
\end{pmatrix}
(mod \text{ }11)
= \begin{pmatrix}
10 \\
7\\
7
\end{pmatrix}
\end{equation}

\begin{equation}
\begin{pmatrix}
 1 & 1 & 0 \\
0 & 0 & 1\\
1 & 0 & 1
\end{pmatrix}
 \times
\begin{pmatrix}
 8\\
5\\
9
\end{pmatrix}
(mod \text{ }11)
= 
\begin{pmatrix}
 13\\
9\\
17
\end{pmatrix}
(mod \text{ }11)
= \begin{pmatrix}
2 \\
9\\
6
\end{pmatrix}
\end{equation}

Декодируем:
\begin{center}
 $ \begin{pmatrix}
5 \\
3\\
7
\end{pmatrix} \to$ \textbf{\textit{ЛЁП}} ;
 $ \begin{pmatrix}
6 \\
6\\
10
\end{pmatrix} \to$ \textbf{\textit{ННЯ}} ;
 $ \begin{pmatrix}
10 \\
7\\
7
\end{pmatrix} \to$ \textbf{\textit{ЯПП}} ;
 $ \begin{pmatrix}
 2 \\
9\\
6
\end{pmatrix} \to$ \textbf{\textit{ДЬН}}  \\

\end{center}
Полученное сообщение:  \textbf{\textit{ЛЁПННЯЯППДЬН}}

\subsection{Шифрование с помощью матрицы-ключа $4 \times 4$}
Матрица-ключ размера  $4 \times 4$ :
\begin{equation}
C =
\begin{pmatrix}
1 & 1 & 0 & 1\\
0 & 0 & 1 & 0 \\
1 & 0 & 1 & 1 \\
1 & 1 & 0 & 0
\end{pmatrix}
\end{equation}
Проверка определителя:
\begin{equation}
\begin{vmatrix}
1 & 1 & 0 & 1\\
0 & 0 & 1 & 0 \\
1 & 0 & 1 & 1 \\
1 & 1 & 0 & 0
\end{vmatrix}
= -1 \neq 0
\end{equation}

Разобьем сообщение на фрагменты по 4 символа и предствим векторы полученных кодов:
\begin{center}
\textbf{\textit{ЗВЁЗ}} $\to \begin{pmatrix}
 4\\
1\\
3\\
4
\end{pmatrix}$ ;

\textbf{\textit{ДНАЯ}}  $\to \begin{pmatrix}
 2\\
6\\
0\\
10
\end{pmatrix}$ ;

\textbf{\textit{ПЫЛЬ}}  $\to \begin{pmatrix}
7\\
8\\
 5\\
9
\end{pmatrix}$ \\
\end{center}
Повторяем действия, описанные в разделе 1.2:
\begin{equation}
\begin{pmatrix}
1 & 1 & 0 & 1\\
0 & 0 & 1 & 0 \\
1 & 0 & 1 & 1 \\
1 & 1 & 0 & 0
\end{pmatrix}
 \times
\begin{pmatrix}
 4\\
1\\
3\\
4
\end{pmatrix}
(mod \text{ }11)
= 
\begin{pmatrix}
9\\
3\\
11\\
5
\end{pmatrix}
(mod \text{ }11)
= \begin{pmatrix}
9\\
3\\
0\\
5
\end{pmatrix}
\end{equation}

\begin{equation}
\begin{pmatrix}
1 & 1 & 0 & 1\\
0 & 0 & 1 & 0 \\
1 & 0 & 1 & 1 \\
1 & 1 & 0 & 0
\end{pmatrix}
 \times
\begin{pmatrix}
 2\\
6\\
0\\
10
\end{pmatrix}
(mod \text{ }11)
= 
\begin{pmatrix}
18\\
0\\
12\\
8
\end{pmatrix}
(mod \text{ }11)
= \begin{pmatrix}
7\\
0\\
1\\
8
\end{pmatrix}
\end{equation}

\begin{equation}
\begin{pmatrix}
1 & 1 & 0 & 1\\
0 & 0 & 1 & 0 \\
1 & 0 & 1 & 1 \\
1 & 1 & 0 & 0
\end{pmatrix}
 \times
\begin{pmatrix}
7\\
8\\
 5\\
9
\end{pmatrix}
(mod \text{ }11)
= 
\begin{pmatrix}
24\\
5\\
21\\
15
\end{pmatrix}
(mod \text{ }11)
= \begin{pmatrix}
2\\
5\\
10\\
4
\end{pmatrix}
\end{equation}

Декодируем:
\begin{center}
 $ \begin{pmatrix}
9\\
3\\
0\\
5
\end{pmatrix} \to$ \textbf{\textit{ЬЁАЛ}} ;
 $ \begin{pmatrix}
7\\
0\\
1\\
8
\end{pmatrix} \to$ \textbf{\textit{ПАВЫ}} ;
 $ \begin{pmatrix}
2\\
5\\
10\\
4
\end{pmatrix} \to$ \textbf{\textit{ДЛЯЗ}} 
 \\

\end{center}
Полученное сообщение:  \textbf{\textit{ЬЁАЛПАВЫДЛЯЗ}}

\subsection{Имитация вредоносного вмешательства}
 a) Повредим фразу, полученную в пункте 1.2
\begin{center}
Таблица 2 -- Повреждение первого результата\\
\begin{tabular}{ |c|c|c|c|c|c|c|c|c|c|c|c|c| } 
 \hline
Исходные символы & Н & Ё & А & З & Ё  & П & Ь  & Д & В  & В & В  & Д\\
\hline
После атаки  & Н & Л & А & З & Ь  & П & Ь  & Д & Ы  & В & В  & Д\\
 \hline
Коды после атаки & 6 & 5 & 0 & 4 & 9  & 7 & 9  & 2 & 8  & 1 & 1  & 2  \\
 \hline
\end{tabular}
\end{center}
Найдем обратную матрицу от первого ключа:
\begin{equation}
A^{-1} =
\begin{pmatrix}
1 & 2 \\
4 & 9
\end{pmatrix} ^{-1}
= 
\begin{pmatrix}
 9 & -2\\
 -4 &  1
\end{pmatrix}
\end{equation}
Разобьем фразу  \textbf{\textit{НЛАЗЬПЬДЫВВД}} на фрагменты:
\begin{center}
\textbf{\textit{НЛ}}
 $ \to \begin{pmatrix}
 6\\
5
\end{pmatrix} $  ;
 \textbf{\textit{АЗ}} $\to \begin{pmatrix}
 0\\
4
\end{pmatrix} $ ;
 \textbf{\textit{ЬП}} $\to \begin{pmatrix}
 9\\
7
\end{pmatrix} $ ;
 \textbf{\textit{ЬД}}  $ \to \begin{pmatrix}
9\\
2
\end{pmatrix}$; \\
\textbf{\textit{ЫВ}} $\to \begin{pmatrix}
 8\\
1
\end{pmatrix} $  ;
 \textbf{\textit{ВД}} $ \to \begin{pmatrix}
 1\\
2
\end{pmatrix}$ \\
\end{center}

Расшифруем сообщение:
\begin{equation}
\begin{pmatrix}
 9 & -2\\
 -4 &  1
\end{pmatrix}
 \times
\begin{pmatrix}
 6\\
5
\end{pmatrix}
(mod \text{ }11)
= 
\begin{pmatrix}
 44\\
-19
\end{pmatrix}
(mod \text{ }11)
= \begin{pmatrix}
 0\\
3
\end{pmatrix}
\end{equation}

\begin{equation}
\begin{pmatrix}
 9 & -2\\
 -4 &  1
\end{pmatrix}
 \times
\begin{pmatrix}
 0\\
4
\end{pmatrix}
(mod \text{ }11)
= 
\begin{pmatrix}
 -8\\
4
\end{pmatrix}
(mod \text{ }11)
= \begin{pmatrix}
3\\
4
\end{pmatrix}
\end{equation}

\begin{equation}
\begin{pmatrix}
 9 & -2\\
 -4 &  1
\end{pmatrix}
 \times
\begin{pmatrix}
 9\\
7
\end{pmatrix}
(mod \text{ }11)
= 
\begin{pmatrix}
 67\\
-29
\end{pmatrix}
(mod \text{ }11)
= \begin{pmatrix}
1\\
4
\end{pmatrix}
\end{equation}

\begin{equation}
\begin{pmatrix}
 9 & -2\\
 -4 &  1
\end{pmatrix}
 \times
\begin{pmatrix}
9\\
2
\end{pmatrix}
(mod \text{ }11)
= 
\begin{pmatrix}
 77\\
-34
\end{pmatrix}
(mod \text{ }11)
= \begin{pmatrix}
0\\
10
\end{pmatrix}
\end{equation}

\begin{equation}
\begin{pmatrix}
 9 & -2\\
 -4 &  1
\end{pmatrix}
 \times
\begin{pmatrix}
 8\\
1
\end{pmatrix}
(mod \text{ }11)
= 
\begin{pmatrix}
 70\\
-31
\end{pmatrix}
(mod \text{ }11)
= \begin{pmatrix}
4\\
2
\end{pmatrix}
\end{equation}


\begin{equation}
\begin{pmatrix}
 9 & -2\\
 -4 &  1
\end{pmatrix}
 \times
\begin{pmatrix}
 1\\
2
\end{pmatrix}
(mod \text{ }11)
= 
\begin{pmatrix}
 5\\
-2
\end{pmatrix}
(mod \text{ }11)
= \begin{pmatrix}
5\\
9
\end{pmatrix}
\end{equation}


Декодируем полученный результат:
\begin{center}
 $ \begin{pmatrix}
0\\
3
\end{pmatrix} \to$ \textbf{\textit{АЁ}} ;
 $ \begin{pmatrix}
3\\
4
\end{pmatrix} \to$ \textbf{\textit{ЁЗ}} ;
 $ \begin{pmatrix}
1\\
4
\end{pmatrix} \to$ \textbf{\textit{ВЗ}} ;
 $ \begin{pmatrix}
0\\
10
\end{pmatrix} \to$ \textbf{\textit{АЯ}} ; \\
 $ \begin{pmatrix}
4\\
2
\end{pmatrix} \to$ \textbf{\textit{ЗД}} ;
$\begin{pmatrix}
5\\
9
\end{pmatrix} \to$ \textbf{\textit{ЛЬ}}  \\
\end{center}
Полученное сообщение:  \textbf{\textit{\colorbox{red! 50}{АЁ}\colorbox{green! 50}{ЁЗ}\colorbox{red! 50}{ВЗ}\colorbox{green! 50}{АЯ}\colorbox{red! 50}{ЗД}\colorbox{green! 50}{ЛЬ}}}\\
Заметим, что поврежденными участками после расшифровки оказались те пары букв, в которых мы провели подмену символов.\\

 б) Повредим фразу, полученную в пункте 1.3
\begin{center}
Таблица 2 -- Повреждение  второго результата\\
\begin{tabular}{ |c|c|c|c|c|c|c|c|c|c|c|c|c| } 
 \hline
Исходные символы & Л & Ё & П & Н & Н  & Я & Я  & П & П  & Д & Ь  & Н\\
\hline
После атаки & Л & Ё & П & Н & Ы  & А & Я  & В & П  & Д & Ь  & Н\\
 \hline
Коды после атаки & 5 & 3 & 7 & 6 & 8  & 0 & 10  & 1 & 7  & 2 & 9  & 6  \\
 \hline
\end{tabular}
\end{center}
Найдем обратную матрицу от второго ключа:
\begin{equation}
B^{-1} = 
\begin{pmatrix}
  1 & 1 & 0 \\
0 & 0 & 1\\
1 & 0 & 1
\end{pmatrix}^{-1}
 = 
\begin{pmatrix}
 0 & -1 & 1 \\
1 & 1 & -1\\
0 & 1 & 0
\end{pmatrix}
\end{equation}
Разобьем фразу  \textbf{\textit{ЛЁПНЫАЯВПДЬН}} на фрагменты:
\begin{center}
\textbf{\textit{ЛЁП}} $\to \begin{pmatrix}
 5\\
3\\
7
\end{pmatrix}$ ;

\textbf{\textit{НЫА}}  $\to \begin{pmatrix}
6\\
8\\
0
\end{pmatrix}$ ;
\textbf{\textit{ЯВП}}  $\to \begin{pmatrix}
 10\\
1\\
7
\end{pmatrix}$ ;

\textbf{\textit{ДЬН}}  $\to \begin{pmatrix}
2\\
9\\
6
\end{pmatrix}$ \\
\end{center}

Расшифруем сообщение:
\begin{equation}
\begin{pmatrix}
 0 & -1 & 1 \\
1 & 1 & -1\\
0 & 1 & 0
\end{pmatrix}
 \times
\begin{pmatrix}
 5\\
3\\
7
\end{pmatrix}
(mod \text{ }11)
= 
\begin{pmatrix}
4\\
1\\
3
\end{pmatrix}
(mod \text{ }11)
= \begin{pmatrix}
4 \\
1\\
3
\end{pmatrix}
\end{equation}

\begin{equation}
\begin{pmatrix}
 0 & -1 & 1 \\
1 & 1 & -1\\
0 & 1 & 0
\end{pmatrix}
 \times
\begin{pmatrix}
6\\
8\\
0
\end{pmatrix}
(mod \text{ }11)
= 
\begin{pmatrix}
-8\\
14\\
8
\end{pmatrix}
(mod \text{ }11)
= \begin{pmatrix}
3 \\
3\\
8
\end{pmatrix}
\end{equation}

\begin{equation}
\begin{pmatrix}
 0 & -1 & 1 \\
1 & 1 & -1\\
0 & 1 & 0
\end{pmatrix}
 \times
\begin{pmatrix}
 10\\
1\\
7
\end{pmatrix}
(mod \text{ }11)
= 
\begin{pmatrix}
6\\
4\\
1
\end{pmatrix}
(mod \text{ }11)
= \begin{pmatrix}
6 \\
4\\
1
\end{pmatrix}
\end{equation}

\begin{equation}
\begin{pmatrix}
 0 & -1 & 1 \\
1 & 1 & -1\\
0 & 1 & 0
\end{pmatrix}
 \times
\begin{pmatrix}
2\\
9\\
6
\end{pmatrix}
(mod \text{ }11)
= 
\begin{pmatrix}
-3\\
5\\
9
\end{pmatrix}
(mod \text{ }11)
= \begin{pmatrix}
8 \\
5\\
9
\end{pmatrix}
\end{equation}

Декодируем полученный результат:
\begin{center}
 $ \begin{pmatrix}
4 \\
1\\
3
\end{pmatrix} \to$ \textbf{\textit{ЗВЁ}} ;
 $ \begin{pmatrix}
3 \\
3\\
8
\end{pmatrix} \to$ \textbf{\textit{ЁЁЫ}} ;
 $ \begin{pmatrix}
6 \\
4\\
1
\end{pmatrix} \to$ \textbf{\textit{НЗВ}} ;
 $ \begin{pmatrix}
8 \\
5\\
9
\end{pmatrix} \to$ \textbf{\textit{ЫЛЬ}}  \\

\end{center}
Полученное сообщение:  \textbf{\textit{\colorbox{green! 50}{ЗВЁ}\colorbox{red! 50}{ЁЁЫ}\colorbox{red! 50}{НЗВ}\colorbox{green! 50}{ЫЛЬ}}}\\
Аналогично предыдущему пункту ошибки проявились только в тех фрагментах, в которых были заменены символы.\\


 в) Повредим фразу, полученную в пункте 1.4
\begin{center}
Таблица 3 -- Повреждение третьего результата\\
\begin{tabular}{ |c|c|c|c|c|c|c|c|c|c|c|c|c| } 
 \hline
Исходные символы & В & Л & П & П & Д  & Д & Н  & Д & Н  & Я & З  & Д\\
\hline
После атаки  & В & Л & Ы & П & Ь  & Д & Н  & Д & Н  & Я & З  & А \\
 \hline
Коды после атаки & 1 & 5 & 8 & 7 & 9  & 2 & 6  & 2 & 6  & 10 & 4  & 0  \\
 \hline
\end{tabular}
\end{center}
Найдем обратную матрицу от третьеого ключа:
\begin{equation}
\begin{pmatrix}
 5 & 3 & 2 & 4 \\
 0 & 5 & 3 & 6 \\
1& 8 & 2 & 0\\
 5 & 7 & 0 & 6
\end{pmatrix}^{-1}
=  \frac{1}{433}\begin{pmatrix}
  84 & -58 & 3 & 2 \\
\\
 -39 & -4 & 45 & 30 \\
\\
114 & 45 & 35 & -121\\
\\
 - \frac{49}{2} & 53 & -55 &  \frac{71}{2}
\end{pmatrix}
\end{equation}

\section{Задание 2. Взлом шифра Хилла}	

\section{Задание 3. Код Хэмминга}

\section{Задание 4. Код Хэмминг?}
\end{document}













