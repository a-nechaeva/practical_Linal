\documentclass[a5paper, 10pt]{article}

% Текст
\usepackage[utf8]{inputenc} % UTF-8 кодировка
\usepackage[russian]{babel} % Русский язык
\usepackage{indentfirst} % красная строка в первом параграфе в главе
% Отображение страниц
\usepackage{geometry} % размеры листа и отступов
\geometry{
	left=12mm,
	top=25mm,
	right=15mm,
	bottom=17mm,
	marginparsep=0mm,
	marginparwidth=0mm,
	headheight=10mm,
	headsep=7mm,
	nofoot}
\usepackage{afterpage,fancyhdr} % настройка колонтитулов
\pagestyle{fancy}
\fancypagestyle{style}{ % создание нового стиля style
	\fancyhf{} % очистка колонтитулов
	\fancyhead[LO, RE]{Лабораторная работа № 1 } % название документа наверху
	\fancyhead[RO, LE]{ Кодирование и шифрование} % название section наверху
	\fancyfoot[RO, LE]{\thepage} % номер страницы справа внизу на нечетных и слева внизу на четных
	\renewcommand{\headrulewidth}{0.25pt} % толщина линии сверху
	\renewcommand{\footrulewidth}{0pt} % толцина линии снизу
}
\fancypagestyle{plain}{ % создание нового стиля plain -- полностью пустого
	\fancyhf{}
	\renewcommand{\headrulewidth}{0pt}
}
\fancypagestyle{title}{ % создание нового стиля title -- для титульной страницы
	\fancyhf{}
	\fancyhead[C]{{\footnotesize
			Министерство образования и науки Российской Федерации\\
			Федеральное государственное автономное образовательное учреждение высшего образования
	}}
	\fancyfoot[C]{{\large 
			Санкт-Петербург, 2023-2024
	}}
	\renewcommand{\headrulewidth}{0pt}
}

% Математика
\usepackage{amsmath, amsfonts, amssymb, amsthm} % Набор пакетов для математических текстов
%\usepackage{dmvnbase} % мехматовский пакет latex-сокращений
\usepackage{cancel} % зачеркивание для сокращений
% Рисунки и фигуры
\usepackage[pdftex]{graphicx} % вставка рисунков
\usepackage{wrapfig, subcaption} % вставка фигур, обтекая текст
\usepackage{caption} % для настройки подписей
\captionsetup{figurewithin=none,labelsep=period, font={small,it}} % настройка подписей к рисункам
% Рисование
\usepackage{tikz} % рисование
\usepackage{circuitikz}
\usepackage{pgfplots} % графики
% Таблицы
\usepackage{multirow} % объединение строк
\usepackage{multicol} % объединение столбцов
% Остальное
\usepackage[unicode, pdftex]{hyperref} % гиперссылки
\usepackage{enumitem} % нормальное оформление списков
\setlist{itemsep=0.15cm,topsep=0.15cm,parsep=1pt} % настройки списков
% Теоремы, леммы, определения...
\theoremstyle{definition}
\newtheorem{Def}{Определение}
\newtheorem*{Axiom}{Аксиома}
\theoremstyle{plain}
\newtheorem{Th}{Теорема}
\newtheorem{Lem}{Лемма}
\newtheorem{Cor}{Следствие}
\newtheorem{Ex}{Пример}
\theoremstyle{remark}
\newtheorem*{Note}{Замечание}
\newtheorem*{Solution}{Решение}
\newtheorem*{Proof}{Доказательство}
% Свои команды
\newcommand{\comb}[1]{\left[\hspace{-4pt}\begin{array}{l}#1\end{array}\right.\hspace{-5pt} } % совокупность уравнений
% Титульный лист
\usepackage{csvsimple-l3}
\newcommand*{\titlePage}{
	\thispagestyle{title}
	\begingroup
	\begin{center}
		%		{\footnotesize
			%			Министерство образования и науки Российской Федерации\\
			%			Федеральное государственное автономное образовательное учреждение высшего образования
			%		}
		%		
		\vspace*{6ex}
		
		{\small
			САНКТ-ПЕТЕРБУРГСКИЙ НАЦИОНАЛЬНЫЙ ИССЛЕДОВАТЕЛЬСКИЙ УНИВЕРСИТЕТ ИТМО	
		}
		
		\vspace*{2ex}
		
		{\normalsize
			Факультет систем управления и робототехники
		}
		
		\vspace*{15ex}
		
		{\Large \bfseries 
			Лабораторная работа № 1
		}
\vspace*{2ex}
	{\Large \bfseries 
			
"Кодирование и шифрование"
		}
\vspace*{2ex}
		
		{\normalsize
			по дисциплине Практическая линейная алгебра
		}

	\end{center}
	\vspace*{20ex}
	\begin{flushright}
		{\large 
			\underline{Выполнила}: студентка гр. \textbf{R3238}\\
			\begin{flushright}
				\textbf{Нечаева А. А.}\\
			\end{flushright}
		}
		
		\vspace*{5ex}
		
		{\large 
			\underline{Преподаватель}: \textit{Перегудин Алексей Алексеевич}
		}
	\end{flushright}	
	\newpage
	\setcounter{page}{1}
	\endgroup}

\begin{document}
	\titlePage
	\pagestyle{style}
\newpage

\section{Задание 1. Шифр Хилла}
\subsection{Задание алфавита и сообщения}


%\caption{\label{tab:canonsummary}Используемый алфавит .}

\begin{center}
Таблица 1 -- Используемый алфавит\\
\begin{tabular}{ |c|c|c|c|c|c| } 
 \hline
Символ & Код & Символ & Код & Символ & Код\\
\hline
А & 0 & З  & 4 & Ы  & 8 \\
 \hline
В & 1 & Л & 5& Ь  & 9  \\
 \hline
Д & 2 & Н & 6& Я  & 10  \\
 \hline
Ё & 3 & П & 7&   &   \\
 \hline
\end{tabular}
\end{center}

Зашифрованное сообщение: \textbf{\textit{ЗВЁЗДНАЯПЫЛЬ}}

Размер алфавита в нашем случае: $$n = 11 $$

У числа  \textbf{11} нет делителей, кроме единицы и самого числа.

\subsection{Шифрование с помощью матрицы-ключа $2 \times 2$}
Матрица A $2 \times 2$ соотвествует ключевому слову: \textbf{\textit{ЛАНЬ}}
\begin{equation}
\begin{vmatrix}
 5 & 0\\
 6 & 9
\end{vmatrix}
= 5 * 9 - 0 * 6 = 45
\end{equation}
Запишем фразу, подлежащую шифрования с помощью кодов символов алфавита и разобьем наше сообщение на векторы. \\
Далее представлены фрагменты сообщения и соотвествующие векторы кодов:
\begin{center}
\textbf{\textit{ЗВ}} $\to \begin{pmatrix}
 4\\
1
\end{pmatrix}$ ;
\textbf{\textit{ЁЗ}}  $\to \begin{pmatrix}
 3\\
4
\end{pmatrix}$ ;
\textbf{\textit{ДН}}  $\to \begin{pmatrix}
 2\\
6
\end{pmatrix}$ ;
\textbf{\textit{АЯ}}  $\to \begin{pmatrix}
 0\\
10
\end{pmatrix}$ \\
\textbf{\textit{ПЫ}}  $\to \begin{pmatrix}
 7\\
8
\end{pmatrix}$ ;
\textbf{\textit{ЛЬ}}  $\to \begin{pmatrix}
 5\\
9
\end{pmatrix}$ \\
\end{center}

Теперь зашифруем сообщение: матрично умножим ключ на каждый вектор и найдем остаток от деления на размер алфавита от результата:
\begin{equation}
\begin{pmatrix}
 5 & 0\\
 6 & 9
\end{pmatrix}
 \times
\begin{pmatrix}
 4\\
1
\end{pmatrix}
(mod \text{ }11)
= 
\begin{pmatrix}
 20\\
33
\end{pmatrix}
(mod \text{ }11)
= \begin{pmatrix}
 9\\
0
\end{pmatrix}
\end{equation}

\begin{equation}
\begin{pmatrix}
 5 & 0\\
 6 & 9
\end{pmatrix}
 \times
\begin{pmatrix}
 3\\
4
\end{pmatrix}
(mod \text{ }11)
= 
\begin{pmatrix}
 15\\
54
\end{pmatrix}
(mod \text{ }11)
= \begin{pmatrix}
 4\\
10
\end{pmatrix}
\end{equation}

\begin{equation}
\begin{pmatrix}
 5 & 0\\
 6 & 9
\end{pmatrix}
 \times
\begin{pmatrix}
 2\\
6
\end{pmatrix}
(mod \text{ }11)
= 
\begin{pmatrix}
 10\\
66
\end{pmatrix}
(mod \text{ }11)
= \begin{pmatrix}
 10\\
0
\end{pmatrix}
\end{equation}

\begin{equation}
\begin{pmatrix}
 5 & 0\\
 6 & 9
\end{pmatrix}
 \times
\begin{pmatrix}
 0\\
10
\end{pmatrix}
(mod \text{ }11)
= 
\begin{pmatrix}
 0\\
90
\end{pmatrix}
(mod \text{ }11)
= \begin{pmatrix}
 0\\
2
\end{pmatrix}
\end{equation}

\begin{equation}
\begin{pmatrix}
 5 & 0\\
 6 & 9
\end{pmatrix}
 \times
\begin{pmatrix}
 7\\
8
\end{pmatrix}
(mod \text{ }11)
= 
\begin{pmatrix}
 35\\
114
\end{pmatrix}
(mod \text{ }11)
= \begin{pmatrix}
 2\\
4
\end{pmatrix}
\end{equation}

\begin{equation}
\begin{pmatrix}
 5 & 0\\
 6 & 9
\end{pmatrix}
 \times
\begin{pmatrix}
 5\\
9
\end{pmatrix}
(mod \text{ }11)
= 
\begin{pmatrix}
 25\\
111
\end{pmatrix}
(mod \text{ }11)
= \begin{pmatrix}
 3\\
1
\end{pmatrix}
\end{equation}

Декодируем полученный результат:
\begin{center}
 $ \begin{pmatrix}
 9\\
0
\end{pmatrix} \to$ \textbf{\textit{ЬА}} ;
 $ \begin{pmatrix}
 4\\
10
\end{pmatrix} \to$ \textbf{\textit{ЗЯ}} ;
 $ \begin{pmatrix}
 10\\
0
\end{pmatrix} \to$ \textbf{\textit{ЯА}} ;
 $ \begin{pmatrix}
 0\\
2
\end{pmatrix} \to$ \textbf{\textit{АД}} ; \\
 $ \begin{pmatrix}
 2\\
4
\end{pmatrix} \to$ \textbf{\textit{ДЗ}} ;
$\begin{pmatrix}
 3\\
1
\end{pmatrix} \to$ \textbf{\textit{ЁВ}}  \\
\end{center}
Полученное сообщение:  \textbf{\textit{ЬАЗЯЯААДДЗЁВ}}


\subsection{Шифрование с помощью матрицы-ключа $3 \times 3$}
Матрица B $3 \times 3$ соотвествует ключевому слову: \textbf{\textit{ВЛАДАНАДЯ}}
\begin{equation}
\begin{vmatrix}
 1 & 5 & 0\\
 2 & 0 & 6 \\
 0 & 2 & 10
\end{vmatrix}
 = -112
\end{equation}

\subsection{Шифрование с помощью матрицы-ключа $2 \times 2$}
Матрица С $4 \times 4$  \\
соотвествует ключевому слову: \textbf{\textit{ЛЁДЗАЛЁНВЫДАЛПАН}}
\begin{equation}
\begin{vmatrix}
 5 & 3 & 2 & 4 \\
 0 & 5 & 3 & 6 \\
1& 8 & 2 & 0\\
 5 & 7 & 0 & 6
\end{vmatrix}
= -866
\end{equation}

\subsection{Шифрование сообщений с помощью шифра Хилла}



\section{Задание 2. Взлом шифра Хилла}	

\section{Задание 3. Код Хэмминга}

\section{Задание 4. Код Хэмминг?}
\end{document}













